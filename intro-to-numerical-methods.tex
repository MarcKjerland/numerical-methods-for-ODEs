\documentclass{article}        
\usepackage{amsmath}

% Japanese support
\usepackage[utf8]{inputenc}
\usepackage[T1]{fontenc}  
\usepackage{CJKutf8}

% Indent first paragraph at new section
\usepackage{indentfirst}

\newcommand{\ud}{\,\mathrm{d}} 
\newcommand{\dt}{\Delta t} 
%\newcommand{\dt}{h} 
\newcommand{\bigO}{\mathcal{O}}
\newcommand{\un}{u_n}
\newcommand{\uni}{u_{n+1}}
%\newcommand{\unhalf}{u_{n+\tfrac{1}{2}}}
\newcommand{\unhalf}{u_{n+1/2}}

% blackboard font 1
\usepackage{bbm} 
% Blackboard "I" for identity matrix
\newcommand{\Id}{\mathbbm{1}}

\title{Intro to numerical methods for ordinary differential equations}
\author{Marc Kjerland}
\date{\today}

\setlength{\parskip}{1ex} % Space between paragraphs

\begin{document}

\maketitle

\section{Introduction}

This handout will be a reference for simple numerical methods for ordinary differential equations (ODEs).

We are interested solving an \emph{initial value problem} for a differentiable function of one variable, $u(t)$.
The values of $u$ are unknown except at a single time, $t=t_0$, but we know
the rate of change of $u$ via the differential equation

\begin{equation}
    %\frac{\ud u}{\ud t} = f(u),
    u'(t) = f(u(t)),
    \label{ode}
\end{equation}

where $ u'(t) = \frac{\ud u}{\ud t}$ and where $f$ is a well-behaved function which is given, possibly based on physical laws, and which depends on the current physical state $u$.
%Note that $f$ does not depend on $t$; it only depends on $u$.

The initial value problem is given as

\begin{equation}
    \left\{\begin{split}
            &u'(t) = f(u(t)), \\
            &u(t_0) = u_0.
    \end{split}\right .
    \label{ivp}
\end{equation}

In general $f$ can also be a function of $t$, for example if there is some external
time-dependent forcing like a solar cycle. For comparison:

\begin{equation}
    %\frac{\ud u}{\ud t} = f(t,u)
    u'(t) = f(t,u(t))
    \label{odet}
\end{equation}

Equation \eqref{odet} is said to be \emph{non-autonomous} and \eqref{ode} is \emph{autonomous}.
For simplicity we'll only consider the autonomous case, where $f$ is independent of $t$,
but the following methods can all be readily
extended to the non-autonomous case.

Note that there is a close connection between the methods
in this paper
and the quadrature methods for finding the area under a curve:
the rectangle method, the trapezoid rule, Simpson's rule, etc.

\subsection{Systems of equations}

All of the methods listed here can be applied to
systems of differential equations:

\begin{equation}
    \left\{\begin{split}
        &u_1' = f_1(u_1,u_2,\ldots,u_n), \\
        &u_2' = f_2(u_1,u_2,\ldots,u_n), \\
        &\ldots \\
        &u_n' = f_n(u_1,u_2,\ldots,u_n).
    \end{split} \right .
\end{equation}

This can be written concisely as 
\begin{equation}
    \vec{u}' = \vec{f}(\vec{u}).
\end{equation}

For simplicity in notation, we can leave off the $\vec{}\ $ and just write $u$ and $f$.


\section{Euler's method}

We can solve the initial value problem \eqref{ivp} on a discretized time domain by replacing the derivative $u'(t)$ with a finite difference:

\begin{equation}
    %\frac{u(t+\dt) - u(t)}{\dt} \approx u'(t) := f(u(t)).
    \frac{u(t+\dt) - u(t)}{\dt} \approx  f(u(t)).
\end{equation}

Moving some terms around we have

\begin{equation}
    u(t+\dt) \approx u(t) + \dt f(u(t))
    \label{utfeuler}
    %\boxed{u(t+\dt) \approx u(t) + \dt f(u(t))}
    %\label{feuler}
    %\tag{Forward Euler}
\end{equation}

Using this, we can find an approximate solution for $u$ recursively,
starting with $u(t_0)$ to compute $u(t_0 + \dt)$,
then using this to compute $u(t_0 + 2\dt)$, and so on.

We'll use the following simplifying notation:
%Assuming the timestep size $\dt$ is uniform, define
$t_n = t_0 + n\dt$ 
and $u_n = u(t_n)$.
Frequently in this handout we'll assume that we know the value of $u_n$ at the current time $t=t_n$,
 and we want to solve for $\uni$ at the next time $t_{n+1}$.
%\begin{equation}
    %\begin{split}
        %t_1 = t_0 + \dt \\
        %t_2 = t_1 + \dt = t_0 + 2\dt \\
        %\ldots \\
        %t_n = t_{n-1} + \dt = t_0 + n\dt \\
        %%t_{n+1} = t_n + \dt = t_0 + (n+1)\dt.
    %\end{split}
%\end{equation}

%and 
%\begin{equation}
    %\begin{split}
        %u_0 = u(t_0) \\
        %u_1 = u(t_1) \\
        %\ldots \\
        %u_n = u(t_n).
        %%\uni = u(t_{n+1}).
    %\end{split}
%\end{equation}

Using this notation we can rewrite \eqref{utfeuler} as:
\begin{equation}
    \boxed{\uni = u_n + \dt f(u_n)}
    \label{feuler}
    \tag{Forward Euler}
\end{equation}

This is known as Euler's method, or more specifically the Forward Euler method, 
named after the 18th Century Swiss mathematician Leonhard Euler.
This is the simplest \emph{explicit} method, where the value at the next time step, $u(t+\dt)$, 
is given explicitly as a function of known values at previous times.
%Later we'll look at an \emph{implicit} method.

\section{Accuracy}

Surely replacing the derivative term with a finite difference will introduce some error into our solution,
and intuitively we expect the error to depend on the size of $\dt$ which is no longer infinitesimal.
Study of the error term is known as the \emph{accuracy} of the method.

How accurate is the Forward Euler method? Let's expand $u(t+\dt)$ in a Taylor series
(See Appendix \ref{a:bigO} and \ref{a:taylor}):

\begin{align*}
    u(t+\dt) &= u(t) + \dt \, u'(t) + \frac{\dt^2}{2} u''(t) + \ldots \\
    {} &= u(t) + \dt \, u'(t) + \bigO(\dt^2).
\end{align*}

Remember that $u'(t) = f(u(t))$, so 

\begin{equation}
    u(t+\dt) - u(t) = \dt f(u(t)) + \bigO(\dt^2).
\end{equation}

The \emph{local truncation error} at each integration step is proportional to $\dt^2$.
Over multiple integration steps these errors will accumulate to a \emph{global truncation error};
since the total number of integration steps is proportional to $\dt^{-1}$,
the global truncation error is proportional to $\dt^2 \cdot \dt^{-1} = \dt$.
So if you want to reduce the errors in your solution by a factor of 2,
you must reduce the step size by a factor of 2, which means you will perform twice as many
calculations.

Are there more accurate methods? Perhaps with a global truncation error of $\bigO(\dt^2)$ or better?
There are, as we'll see in the next section. A method with error $\bigO(\dt^n)$ 
is said to be \emph{order-n} accurate or \emph{nth-order} accurate.
The Forward Euler method is \emph{order-1} accurate.

\section{Higher order methods}

There is a family of numerical methods which use multiple
values of $u'$ between $t$ and $t+\dt$ to get a more accurate estimate for $u(t+\dt)$.
These methods (just like higher-order quadrature methods) require only that the higher derivates 
of $f$ are bounded.

The first example is the \emph{midpoint method}, a 2nd-order method.

\subsection{Midpoint method}

The basic idea of the midpoint method is that the finite difference should be more accurate
if it is centered around the derivative:

\begin{equation}
    %\frac{u(t+\dt) - u(t)}{\dt} \approx u'(t+\tfrac{\dt}{2}).
    \frac{u(t+\dt) - u(t)}{\dt} \approx u'(t+\frac{\dt}{2}).
\end{equation}

This can be verified by replacing $u(t+\dt)$ and $u(t)$ with Taylor expansions of $u$ around $t+\frac{\dt}{2}$:

\begin{align*}
    u(t+\dt) &= u(t+\frac{\dt}{2}) + \frac{\dt}{2} u'(t+\frac{\dt}{2}) + \frac{\dt^2}{8} u''(t+\frac{\dt}{2}) + 
    \bigO(\dt^3) \\
    %\frac{\dt^3}{48} u'''(t+\frac{\dt}{2}) + \ldots \\ %\bigO(\dt^4) \\
    u(t) &= u(t+\frac{\dt}{2}) - \frac{\dt}{2} u'(t+\frac{\dt}{2}) + \frac{\dt^2}{8} u''(t+\frac{\dt}{2}) + 
    \bigO(\dt^3) \\
    %\frac{\dt^3}{48} u'''(t+\frac{\dt}{2}) + \ldots \\ %\bigO(\dt^4) \\
    \implies u(t+\dt) - u(t) &= \dt\, u'(t+\frac{\dt}{2}) + \bigO(\dt^3).
\end{align*}

So, if we apply this to our differential equation \eqref{ode}, we have

\begin{equation}
    %u(t+\dt) - u(t) = \dt f(u(t+\frac{\dt}{2})) + \bigO(\dt^3).
    \uni = u_n + \dt\, f(\unhalf) + \bigO(\dt^3).
    \label{mid-error}
\end{equation}

%Unfortunately we don't know the value of $u(t+\frac{\dt}{2})$. 
Unfortunately we don't know the value of $\unhalf$. 
However, we know how to approximate it with a Taylor expansion:

\begin{equation}
    \begin{split}
        u(t+\frac{\dt}{2}) &= u(t) + \frac{\dt}{2} u'(t) + \bigO(\dt^2) \\
        %{} &= u(t) + \frac{\dt}{2} f(u(t)) + \bigO(\dt^2).
        \implies \unhalf &= u_n + \frac{\dt}{2} f(u_n) + \bigO(\dt^2).
    \end{split}
    \label{mid-insides}
\end{equation}

Putting this all together we have:
\begin{equation}
    %\boxed{u(t+\dt) = u(t) + \dt f\left(u(t) + \frac{\dt}{2} f(u(t))\right)}
    \boxed{\uni = u_n + \dt\, f\left(u_n + \tfrac{\dt}{2} f(u_n)\right)}
    \tag{Midpoint method}
\end{equation}

From \eqref{mid-error} we see that the midpoint method has local error of order $\dt^3$,
and as long as $f$ is differentiable the remainder term in \eqref{mid-insides} won't
result in larger errors. Thus the midpoint method has a \emph{global} error of $\dt^2$,
so we say it is a 2nd-order method.

This accuracy comes at a cost: for each integration step that the Forward Euler method would take,
the midpoint method requires two integration steps:
%first to get $u(t + \frac{\dt}{2})$ then to get $u(t + \dt)$.
first to find $\unhalf$ then to find $\uni$.
However, this is a small price to pay for the increased order of accuracy.

\subsection{Runge-Kutta methods}

Following the example of the midpoint method, we can add more and more intermediate terms
to our calculations to increase the order of accuracy. Two German mathematicians, C. Runge and M.W. Kutta,
explored these methods around the year 1900 and now they are known as the Runge-Kutta methods.
The midpoint method is a simple case of a Runge-Kutta method and is sometimes referred to as RK2.

One of the most popular Runge-Kutta methods is 4th-order and requires 
four calculations at each integration step. It is often referred to as RK4:

\begin{equation}
    \boxed{\begin{split}
        &\uni = u_n + \frac{\dt}{6} (k_1 + k_2 + k_3 + k_4) \\
        &k_1 = f(u_n) \\ 
        &k_2 = f(u_n + \frac{\dt}{2} k_1) \\ 
        &k_3 = f(u_n + \frac{\dt}{2} k_2) \\ 
        &k_4 = f(u_n + \dt k_3).
        %&u(t+\dt) = u(t) + \frac{\dt}{6} (k_1 + k_2 + k_3 + k_4) \\
        %&k_1 = f(u(t)) \\ 
        %&k_2 = f(u(t) + \frac{\dt}{2} k_1) \\ 
        %&k_3 = f(u(t) + \frac{\dt}{2} k_2) \\ 
        %&k_4 = f(u(t) + \dt k_3).
    \end{split}}
    \label{rk4}
    \tag{RK4}
\end{equation}

%There are also some \emph{time-adaptive} Runge-Kutta methods which allow for clever prediction
%of the error

Matlab's standard ODE solver \texttt{ode45} is a 4th order Runge-Kutta solver
(but it isn't RK4: it also uses an adaptive timestep feature).

\section{Stability}

Sometimes a differential equation is especially challenging to solve numerically,
and if your timestep isn't small enough your solution will be very wrong.
This problem is known as \emph{stability}, and particularly troublesome
equations are sometimes called "stiff."

Here is a very simple example which turns out to be a stiff problem.
Consider a linear drag law, or Stokes' law,
where the frictional force $F_d$ on an object or a fluid parcel is proportional to the velocity $u$:

\begin{equation}
    F_d = -\alpha u,
    \label{stokes}
\end{equation}

for some $\alpha > 0$.
For an object with unit mass, by Newton's second law the acceleration $\tfrac{\ud u}{\ud t}$ is 
given by:

\begin{equation}
    \frac{\ud u}{\ud t} = -\alpha u.
\end{equation}

This is very simple to solve analytically [what is the solution?]
but let's apply the Forward Euler method:

\begin{equation}
    \begin{aligned}
        \frac{\uni - u_n}{\dt} &= -\alpha u_n \\
        \implies \uni &= u_n - \dt\,\alpha u_n \\
                 {} &= (1 - \dt\,\alpha) u_n.
    \end{aligned}
    \label{dragexp}
\end{equation}

If the timestep $\dt$ is small enough, we'll get the result we expect.
But what happens if the drag coefficient $\alpha$ is large, or $\dt$ is large,
so that $\dt \geq \alpha^{-1}$?

[Try an example! $u_n = 5, \alpha = 10, \dt = 0.5$]

When $\dt > \alpha^{-1}$ the method becomes \emph{unstable} and the solutions
will be unphysical.

The stability of a numerical method depends strongly on the differential equation itself,
and this issue is especially important for numerical solution of partial differential equations (PDEs).

\subsection{CFL condition}

In fluid dynamics there is an important stability criteria known as the \emph{CFL condition}
(Courant, Friedrichs, Lewy), sometimes called the Courant condition.
It specifies an upper limit for the timestep size $\Delta t$
based on the spatial grid size $\Delta x$ and the speed $C$ of the fastest 
characteristic wave:

\begin{equation}
    \Delta t \leq \frac{\Delta x}{C}
    \tag{CFL}
\end{equation}

In an explicit numerical scheme, this condition must be satisfied or 
the solution will be unphysical.

For example: long water waves propagate with speed $u \pm \sqrt{gh}$, where
$u$ is the water current speed and $h$ is the height of the water column. So we would take
$C = \max_x (|u| + \sqrt{gh})$ to find a constraint for the timestep.
This means we may have to take smaller time steps if there
are areas of fast flow, high waves, or deeper bathymetry.

Another example is the speed of sound in a compressible gas, which is much higher
than the speed of gravity waves; this is one reason
why the compressible Euler equations are not used in meteorology.

The CFL condition creates a costly problem - if you want fine spatial resolution you will reduce $\Delta x$,
but by the CFL conditon you must also reduce $\Delta t$. [See: Curse of Dimensionality]

\section{Implicit methods}

Stability constraints can sometimes be severe, requiring $\dt$ to be so small that
your numerical solution is very costly to compute. One way to avoid this is to use an
\emph{implicit} method. 

Going back to our original ODE \eqref{ode}, recall we replaced the derivative on the left hand side
with a difference equation from $u_n$ to $\uni$ and we used $u_n$ on the right hand side, like so:

\begin{equation}
    \frac{\uni - u_n }{ \dt } = f(u_n).
\end{equation}

What if we instead used $\uni$ in the right hand side?

\begin{equation}
    \frac{\uni - u_n }{ \dt } = f(\uni).
\end{equation}

Rearranging the terms we have

\begin{equation}
    \boxed{\uni + \dt\, f(\uni) = u_n}
    \label{beuler}
    \tag{Backward Euler}
\end{equation}

Now we would have to solve an algebraic equation for $\uni$, and often this is done
using an iterative solution method such as Newton's method.

For a linear system the Backward Euler method is much simpler:
\begin{equation}
    u' = A u
    \label{linode}
\end{equation}

where $A$ is a scalar or matrix. Then we have

\begin{equation}
    \begin{split}
        &\frac{\uni - u_n}{\dt} = A \uni \\
        \implies &\uni - \dt A \uni = u_n \\
        \implies &\left(\Id - \dt A \right) \uni = u_n \\
        \implies  &\uni = \left(\Id - \dt A \right)^{-1}  u_n, \\
    \end{split}
\end{equation}

where $\Id$ is an identity matrix (or $1$ in the scalar case).


Thus to solve the linear equation \eqref{linode} the Backwards Euler method is given by

\begin{equation}
    \boxed{\uni = (\Id - \dt A)^{-1} u_n}
    \label{linbeuler}
    \tag{Backward Euler, linear}
\end{equation}

The Backward Euler method is the simplest implicit method, and it has the same accuracy
as the Forward Euler method (order 1). Higher order implicit methods exist,
as well as hybrid implicit-explicit methods.

Implicit methods always require some extra calculation steps
such as iteration or matrix inversion*
which in practice can be costly. 
However, they are invaluable for certain problems
(as we'll see in the next example).

[*Here is an excellent essay why you should never actually invert a matrix:
http://www.johndcook.com/blog/2010/01/19/dont-invert-that-matrix/]

Note 1: The Backward Euler method
will fail if $(\Id - \dt A)$ is not invertible, but in practice this
is usually not a problem (see what happens in the next example).

Note 2: In the nonlinear case, an approximation can be made to $f$ at each time step
such as replacing it locally with with a tangent line. Then linear algebra can be used
to get a solution.


\subsection{Application to stiff problem}

Implicit methods are best suited for stiff problems.
Let's return to the scalar ODE for linear drag law (Stokes law):

\begin{equation}
    \frac{\ud u}{\ud t} = -\alpha u
\end{equation}

and now let's apply the Backward Euler method:

\begin{equation}
    \begin{split}
        &\frac{\uni - u_n}{\dt} = -\alpha \uni \\
        \implies &\uni = u_n - \dt\,\alpha \uni \\
        \implies &\uni + \dt\,\alpha \uni = u_n \\
        \implies &(1 + \dt\,\alpha )\uni = u_n \\
        \implies &\uni = \frac{u_n}{1 + \dt\,\alpha }.
    \end{split}
\end{equation}

Compare this with the explicit case \eqref{dragexp}.
In this implicit case, the term $(1 + \dt\,\alpha)$ is always positive
so $\uni$ will always be the same sign as $u_n$, and this will drive
the solution towards zero as expected. In fact, $\dt$ could be quite
large and it will still give a qualitatively correct result
(although the accuracy will suffer).

So, implicit methods help ignore stability constraints
and thus allow for larger $\dt$, resulting in fewer timesteps 
in total. The trade-off is that each timestep will be more costly,
especially with high-dimensional systems,
and the cost also usually increases with the size of $\dt$.

%\section{Which method to use?}


%\section{PDEs - Finite Difference Method}

%\subsection{Action Balance Equations}

%\section{PDEs - Finite Volume Method}

%\subsection{Shallow Water Equations}

%\section{PDEs - Finite Element Method}

\appendix

\section{"Big Oh" notation}
\label{a:bigO}

The "Big Oh" notation is often used in
applied math and computer science
when discussing roughly how large (or small) a function is depending on some variable.
You can think of it like an Order of Magnitude.
In this handout we are specifically interested how a small parameter like $\dt$
affects the error or remainder term in our approximations.

We say $g(h)=\bigO(h)$ if there is a finite constant $C$ such that
$\frac{g(h)}{h} \rightarrow C \neq 0$ 
as $h \rightarrow 0$.
So roughly $g(h) \approx C h$ for small values of $h$.

Example. Let $g(h) = -6h^2 + 25 h^3 + 100 h^4$.
We are interested in values of $h$ which are small (close to 0). In that case, $|h^4| << |h^3| << |h^2|$.
So for small $h$, $g(h) \approx -6h^2.$ Then we say $g(h) = \bigO(h^2)$.

In this handout if you see an expression like $\bigO(\dt^3)$, that means there is some
error/remainder term which might not be known exactly but which is roughly the size of $\dt^3$.

In computer science, big-oh notation is usually used to describe
how long an algorithm takes based on the problem size, represented by a large integer $n$,
such as matrix multiplication or the famous \emph{P vs NP} question.

Trivia (1): Multiplying two $n\times n$ matrices takes $\bigO(n^3)$ steps using the basic method,
but the best algorithm takes $\bigO(n^{2.8})$ steps.
For $1000\times 1000$ matrices, in theory that's nearly four times faster!

Trivia (2): Computing the discrete Fourier transform of $n$ data points
based on the definition takes $\bigO(n^2)$ steps,
but the Fast Fourier Transform does it in $\bigO(n \log n)$ steps.
For 1000 data points that's a speedup of 100x !!

\section{Taylor series expansion}
\label{a:taylor}
Taylor's Theorem tells us that we can represent a function $g(x)$
by its Taylor series expansion

\begin{equation}
    g(x+h) = g(x) + h g'(x) + \frac{h^2}{2} g''(x) + \frac{h^3}{6} g'''(x) + \ldots
\end{equation}

within a small distance $h$ from $x$.
Furthermore, if $g$ is nice (its higher derivatives aren't too big) then we can truncate the Taylor series
and get a good approximation:

\begin{align*}
    &g(x+h) \approx g(x) + h g'(x) &+ \bigO(h^2) \\
    &g(x+h) \approx g(x) + h g'(x) + \frac{h^2}{2} g''(x) &+ \bigO(h^3) \\
    &g(x+h) \approx g(x) + h g'(x) + \frac{h^2}{2} g''(x) + \frac{h^3}{6} g'''(x) &+ \bigO(h^4)
\end{align*}

and so on, where the size of the error is roughly a power of $h$ as $h\rightarrow 0$.

\section{Pronounciation guide}

Euler: \begin{CJK}{UTF8}{min}オイラー\end{CJK}

Runge-Kutta: \begin{CJK}{UTF8}{min}ルンゲ クッタ\end{CJK}

\end{document}



